\documentclass[11pt]{article}

    \usepackage[breakable]{tcolorbox}
    \usepackage{parskip} % Stop auto-indenting (to mimic markdown behaviour)
    

    % Basic figure setup, for now with no caption control since it's done
    % automatically by Pandoc (which extracts ![](path) syntax from Markdown).
    \usepackage{graphicx}
    % Maintain compatibility with old templates. Remove in nbconvert 6.0
    \let\Oldincludegraphics\includegraphics
    % Ensure that by default, figures have no caption (until we provide a
    % proper Figure object with a Caption API and a way to capture that
    % in the conversion process - todo).
    \usepackage{caption}
    \DeclareCaptionFormat{nocaption}{}
    \captionsetup{format=nocaption,aboveskip=0pt,belowskip=0pt}

    \usepackage{float}
    \floatplacement{figure}{H} % forces figures to be placed at the correct location
    \usepackage{xcolor} % Allow colors to be defined
    \usepackage{enumerate} % Needed for markdown enumerations to work
    \usepackage{geometry} % Used to adjust the document margins
    \usepackage{amsmath} % Equations
    \usepackage{amssymb} % Equations
    \usepackage{textcomp} % defines textquotesingle
    % Hack from http://tex.stackexchange.com/a/47451/13684:
    \AtBeginDocument{%
        \def\PYZsq{\textquotesingle}% Upright quotes in Pygmentized code
    }
    \usepackage{upquote} % Upright quotes for verbatim code
    \usepackage{eurosym} % defines \euro

    \usepackage{iftex}
    \ifPDFTeX
        \usepackage[T1]{fontenc}
        \IfFileExists{alphabeta.sty}{
              \usepackage{alphabeta}
          }{
              \usepackage[mathletters]{ucs}
              \usepackage[utf8x]{inputenc}
          }
    \else
        \usepackage{fontspec}
        \usepackage{unicode-math}
    \fi

    \usepackage{fancyvrb} % verbatim replacement that allows latex
    \usepackage{grffile} % extends the file name processing of package graphics
                         % to support a larger range
    \makeatletter % fix for old versions of grffile with XeLaTeX
    \@ifpackagelater{grffile}{2019/11/01}
    {
      % Do nothing on new versions
    }
    {
      \def\Gread@@xetex#1{%
        \IfFileExists{"\Gin@base".bb}%
        {\Gread@eps{\Gin@base.bb}}%
        {\Gread@@xetex@aux#1}%
      }
    }
    \makeatother
    \usepackage[Export]{adjustbox} % Used to constrain images to a maximum size
    \adjustboxset{max size={0.9\linewidth}{0.9\paperheight}}

    % The hyperref package gives us a pdf with properly built
    % internal navigation ('pdf bookmarks' for the table of contents,
    % internal cross-reference links, web links for URLs, etc.)
    \usepackage{hyperref}
    % The default LaTeX title has an obnoxious amount of whitespace. By default,
    % titling removes some of it. It also provides customization options.
    \usepackage{titling}
    \usepackage{longtable} % longtable support required by pandoc >1.10
    \usepackage{booktabs}  % table support for pandoc > 1.12.2
    \usepackage{array}     % table support for pandoc >= 2.11.3
    \usepackage{calc}      % table minipage width calculation for pandoc >= 2.11.1
    \usepackage[inline]{enumitem} % IRkernel/repr support (it uses the enumerate* environment)
    \usepackage[normalem]{ulem} % ulem is needed to support strikethroughs (\sout)
                                % normalem makes italics be italics, not underlines
    \usepackage{soul}      % strikethrough (\st) support for pandoc >= 3.0.0
    \usepackage{mathrsfs}
    

    
    % Colors for the hyperref package
    \definecolor{urlcolor}{rgb}{0,.145,.698}
    \definecolor{linkcolor}{rgb}{.71,0.21,0.01}
    \definecolor{citecolor}{rgb}{.12,.54,.11}

    % ANSI colors
    \definecolor{ansi-black}{HTML}{3E424D}
    \definecolor{ansi-black-intense}{HTML}{282C36}
    \definecolor{ansi-red}{HTML}{E75C58}
    \definecolor{ansi-red-intense}{HTML}{B22B31}
    \definecolor{ansi-green}{HTML}{00A250}
    \definecolor{ansi-green-intense}{HTML}{007427}
    \definecolor{ansi-yellow}{HTML}{DDB62B}
    \definecolor{ansi-yellow-intense}{HTML}{B27D12}
    \definecolor{ansi-blue}{HTML}{208FFB}
    \definecolor{ansi-blue-intense}{HTML}{0065CA}
    \definecolor{ansi-magenta}{HTML}{D160C4}
    \definecolor{ansi-magenta-intense}{HTML}{A03196}
    \definecolor{ansi-cyan}{HTML}{60C6C8}
    \definecolor{ansi-cyan-intense}{HTML}{258F8F}
    \definecolor{ansi-white}{HTML}{C5C1B4}
    \definecolor{ansi-white-intense}{HTML}{A1A6B2}
    \definecolor{ansi-default-inverse-fg}{HTML}{FFFFFF}
    \definecolor{ansi-default-inverse-bg}{HTML}{000000}

    % common color for the border for error outputs.
    \definecolor{outerrorbackground}{HTML}{FFDFDF}

    % commands and environments needed by pandoc snippets
    % extracted from the output of `pandoc -s`
    \providecommand{\tightlist}{%
      \setlength{\itemsep}{0pt}\setlength{\parskip}{0pt}}
    \DefineVerbatimEnvironment{Highlighting}{Verbatim}{commandchars=\\\{\}}
    % Add ',fontsize=\small' for more characters per line
    \newenvironment{Shaded}{}{}
    \newcommand{\KeywordTok}[1]{\textcolor[rgb]{0.00,0.44,0.13}{\textbf{{#1}}}}
    \newcommand{\DataTypeTok}[1]{\textcolor[rgb]{0.56,0.13,0.00}{{#1}}}
    \newcommand{\DecValTok}[1]{\textcolor[rgb]{0.25,0.63,0.44}{{#1}}}
    \newcommand{\BaseNTok}[1]{\textcolor[rgb]{0.25,0.63,0.44}{{#1}}}
    \newcommand{\FloatTok}[1]{\textcolor[rgb]{0.25,0.63,0.44}{{#1}}}
    \newcommand{\CharTok}[1]{\textcolor[rgb]{0.25,0.44,0.63}{{#1}}}
    \newcommand{\StringTok}[1]{\textcolor[rgb]{0.25,0.44,0.63}{{#1}}}
    \newcommand{\CommentTok}[1]{\textcolor[rgb]{0.38,0.63,0.69}{\textit{{#1}}}}
    \newcommand{\OtherTok}[1]{\textcolor[rgb]{0.00,0.44,0.13}{{#1}}}
    \newcommand{\AlertTok}[1]{\textcolor[rgb]{1.00,0.00,0.00}{\textbf{{#1}}}}
    \newcommand{\FunctionTok}[1]{\textcolor[rgb]{0.02,0.16,0.49}{{#1}}}
    \newcommand{\RegionMarkerTok}[1]{{#1}}
    \newcommand{\ErrorTok}[1]{\textcolor[rgb]{1.00,0.00,0.00}{\textbf{{#1}}}}
    \newcommand{\NormalTok}[1]{{#1}}

    % Additional commands for more recent versions of Pandoc
    \newcommand{\ConstantTok}[1]{\textcolor[rgb]{0.53,0.00,0.00}{{#1}}}
    \newcommand{\SpecialCharTok}[1]{\textcolor[rgb]{0.25,0.44,0.63}{{#1}}}
    \newcommand{\VerbatimStringTok}[1]{\textcolor[rgb]{0.25,0.44,0.63}{{#1}}}
    \newcommand{\SpecialStringTok}[1]{\textcolor[rgb]{0.73,0.40,0.53}{{#1}}}
    \newcommand{\ImportTok}[1]{{#1}}
    \newcommand{\DocumentationTok}[1]{\textcolor[rgb]{0.73,0.13,0.13}{\textit{{#1}}}}
    \newcommand{\AnnotationTok}[1]{\textcolor[rgb]{0.38,0.63,0.69}{\textbf{\textit{{#1}}}}}
    \newcommand{\CommentVarTok}[1]{\textcolor[rgb]{0.38,0.63,0.69}{\textbf{\textit{{#1}}}}}
    \newcommand{\VariableTok}[1]{\textcolor[rgb]{0.10,0.09,0.49}{{#1}}}
    \newcommand{\ControlFlowTok}[1]{\textcolor[rgb]{0.00,0.44,0.13}{\textbf{{#1}}}}
    \newcommand{\OperatorTok}[1]{\textcolor[rgb]{0.40,0.40,0.40}{{#1}}}
    \newcommand{\BuiltInTok}[1]{{#1}}
    \newcommand{\ExtensionTok}[1]{{#1}}
    \newcommand{\PreprocessorTok}[1]{\textcolor[rgb]{0.74,0.48,0.00}{{#1}}}
    \newcommand{\AttributeTok}[1]{\textcolor[rgb]{0.49,0.56,0.16}{{#1}}}
    \newcommand{\InformationTok}[1]{\textcolor[rgb]{0.38,0.63,0.69}{\textbf{\textit{{#1}}}}}
    \newcommand{\WarningTok}[1]{\textcolor[rgb]{0.38,0.63,0.69}{\textbf{\textit{{#1}}}}}


    % Define a nice break command that doesn't care if a line doesn't already
    % exist.
    \def\br{\hspace*{\fill} \\* }
    % Math Jax compatibility definitions
    \def\gt{>}
    \def\lt{<}
    \let\Oldtex\TeX
    \let\Oldlatex\LaTeX
    \renewcommand{\TeX}{\textrm{\Oldtex}}
    \renewcommand{\LaTeX}{\textrm{\Oldlatex}}
    % Document parameters
    % Document title
    \title{GeneracionLibrosCodificacion}
    
    
    
    
    
    
    
% Pygments definitions
\makeatletter
\def\PY@reset{\let\PY@it=\relax \let\PY@bf=\relax%
    \let\PY@ul=\relax \let\PY@tc=\relax%
    \let\PY@bc=\relax \let\PY@ff=\relax}
\def\PY@tok#1{\csname PY@tok@#1\endcsname}
\def\PY@toks#1+{\ifx\relax#1\empty\else%
    \PY@tok{#1}\expandafter\PY@toks\fi}
\def\PY@do#1{\PY@bc{\PY@tc{\PY@ul{%
    \PY@it{\PY@bf{\PY@ff{#1}}}}}}}
\def\PY#1#2{\PY@reset\PY@toks#1+\relax+\PY@do{#2}}

\@namedef{PY@tok@w}{\def\PY@tc##1{\textcolor[rgb]{0.73,0.73,0.73}{##1}}}
\@namedef{PY@tok@c}{\let\PY@it=\textit\def\PY@tc##1{\textcolor[rgb]{0.24,0.48,0.48}{##1}}}
\@namedef{PY@tok@cp}{\def\PY@tc##1{\textcolor[rgb]{0.61,0.40,0.00}{##1}}}
\@namedef{PY@tok@k}{\let\PY@bf=\textbf\def\PY@tc##1{\textcolor[rgb]{0.00,0.50,0.00}{##1}}}
\@namedef{PY@tok@kp}{\def\PY@tc##1{\textcolor[rgb]{0.00,0.50,0.00}{##1}}}
\@namedef{PY@tok@kt}{\def\PY@tc##1{\textcolor[rgb]{0.69,0.00,0.25}{##1}}}
\@namedef{PY@tok@o}{\def\PY@tc##1{\textcolor[rgb]{0.40,0.40,0.40}{##1}}}
\@namedef{PY@tok@ow}{\let\PY@bf=\textbf\def\PY@tc##1{\textcolor[rgb]{0.67,0.13,1.00}{##1}}}
\@namedef{PY@tok@nb}{\def\PY@tc##1{\textcolor[rgb]{0.00,0.50,0.00}{##1}}}
\@namedef{PY@tok@nf}{\def\PY@tc##1{\textcolor[rgb]{0.00,0.00,1.00}{##1}}}
\@namedef{PY@tok@nc}{\let\PY@bf=\textbf\def\PY@tc##1{\textcolor[rgb]{0.00,0.00,1.00}{##1}}}
\@namedef{PY@tok@nn}{\let\PY@bf=\textbf\def\PY@tc##1{\textcolor[rgb]{0.00,0.00,1.00}{##1}}}
\@namedef{PY@tok@ne}{\let\PY@bf=\textbf\def\PY@tc##1{\textcolor[rgb]{0.80,0.25,0.22}{##1}}}
\@namedef{PY@tok@nv}{\def\PY@tc##1{\textcolor[rgb]{0.10,0.09,0.49}{##1}}}
\@namedef{PY@tok@no}{\def\PY@tc##1{\textcolor[rgb]{0.53,0.00,0.00}{##1}}}
\@namedef{PY@tok@nl}{\def\PY@tc##1{\textcolor[rgb]{0.46,0.46,0.00}{##1}}}
\@namedef{PY@tok@ni}{\let\PY@bf=\textbf\def\PY@tc##1{\textcolor[rgb]{0.44,0.44,0.44}{##1}}}
\@namedef{PY@tok@na}{\def\PY@tc##1{\textcolor[rgb]{0.41,0.47,0.13}{##1}}}
\@namedef{PY@tok@nt}{\let\PY@bf=\textbf\def\PY@tc##1{\textcolor[rgb]{0.00,0.50,0.00}{##1}}}
\@namedef{PY@tok@nd}{\def\PY@tc##1{\textcolor[rgb]{0.67,0.13,1.00}{##1}}}
\@namedef{PY@tok@s}{\def\PY@tc##1{\textcolor[rgb]{0.73,0.13,0.13}{##1}}}
\@namedef{PY@tok@sd}{\let\PY@it=\textit\def\PY@tc##1{\textcolor[rgb]{0.73,0.13,0.13}{##1}}}
\@namedef{PY@tok@si}{\let\PY@bf=\textbf\def\PY@tc##1{\textcolor[rgb]{0.64,0.35,0.47}{##1}}}
\@namedef{PY@tok@se}{\let\PY@bf=\textbf\def\PY@tc##1{\textcolor[rgb]{0.67,0.36,0.12}{##1}}}
\@namedef{PY@tok@sr}{\def\PY@tc##1{\textcolor[rgb]{0.64,0.35,0.47}{##1}}}
\@namedef{PY@tok@ss}{\def\PY@tc##1{\textcolor[rgb]{0.10,0.09,0.49}{##1}}}
\@namedef{PY@tok@sx}{\def\PY@tc##1{\textcolor[rgb]{0.00,0.50,0.00}{##1}}}
\@namedef{PY@tok@m}{\def\PY@tc##1{\textcolor[rgb]{0.40,0.40,0.40}{##1}}}
\@namedef{PY@tok@gh}{\let\PY@bf=\textbf\def\PY@tc##1{\textcolor[rgb]{0.00,0.00,0.50}{##1}}}
\@namedef{PY@tok@gu}{\let\PY@bf=\textbf\def\PY@tc##1{\textcolor[rgb]{0.50,0.00,0.50}{##1}}}
\@namedef{PY@tok@gd}{\def\PY@tc##1{\textcolor[rgb]{0.63,0.00,0.00}{##1}}}
\@namedef{PY@tok@gi}{\def\PY@tc##1{\textcolor[rgb]{0.00,0.52,0.00}{##1}}}
\@namedef{PY@tok@gr}{\def\PY@tc##1{\textcolor[rgb]{0.89,0.00,0.00}{##1}}}
\@namedef{PY@tok@ge}{\let\PY@it=\textit}
\@namedef{PY@tok@gs}{\let\PY@bf=\textbf}
\@namedef{PY@tok@gp}{\let\PY@bf=\textbf\def\PY@tc##1{\textcolor[rgb]{0.00,0.00,0.50}{##1}}}
\@namedef{PY@tok@go}{\def\PY@tc##1{\textcolor[rgb]{0.44,0.44,0.44}{##1}}}
\@namedef{PY@tok@gt}{\def\PY@tc##1{\textcolor[rgb]{0.00,0.27,0.87}{##1}}}
\@namedef{PY@tok@err}{\def\PY@bc##1{{\setlength{\fboxsep}{\string -\fboxrule}\fcolorbox[rgb]{1.00,0.00,0.00}{1,1,1}{\strut ##1}}}}
\@namedef{PY@tok@kc}{\let\PY@bf=\textbf\def\PY@tc##1{\textcolor[rgb]{0.00,0.50,0.00}{##1}}}
\@namedef{PY@tok@kd}{\let\PY@bf=\textbf\def\PY@tc##1{\textcolor[rgb]{0.00,0.50,0.00}{##1}}}
\@namedef{PY@tok@kn}{\let\PY@bf=\textbf\def\PY@tc##1{\textcolor[rgb]{0.00,0.50,0.00}{##1}}}
\@namedef{PY@tok@kr}{\let\PY@bf=\textbf\def\PY@tc##1{\textcolor[rgb]{0.00,0.50,0.00}{##1}}}
\@namedef{PY@tok@bp}{\def\PY@tc##1{\textcolor[rgb]{0.00,0.50,0.00}{##1}}}
\@namedef{PY@tok@fm}{\def\PY@tc##1{\textcolor[rgb]{0.00,0.00,1.00}{##1}}}
\@namedef{PY@tok@vc}{\def\PY@tc##1{\textcolor[rgb]{0.10,0.09,0.49}{##1}}}
\@namedef{PY@tok@vg}{\def\PY@tc##1{\textcolor[rgb]{0.10,0.09,0.49}{##1}}}
\@namedef{PY@tok@vi}{\def\PY@tc##1{\textcolor[rgb]{0.10,0.09,0.49}{##1}}}
\@namedef{PY@tok@vm}{\def\PY@tc##1{\textcolor[rgb]{0.10,0.09,0.49}{##1}}}
\@namedef{PY@tok@sa}{\def\PY@tc##1{\textcolor[rgb]{0.73,0.13,0.13}{##1}}}
\@namedef{PY@tok@sb}{\def\PY@tc##1{\textcolor[rgb]{0.73,0.13,0.13}{##1}}}
\@namedef{PY@tok@sc}{\def\PY@tc##1{\textcolor[rgb]{0.73,0.13,0.13}{##1}}}
\@namedef{PY@tok@dl}{\def\PY@tc##1{\textcolor[rgb]{0.73,0.13,0.13}{##1}}}
\@namedef{PY@tok@s2}{\def\PY@tc##1{\textcolor[rgb]{0.73,0.13,0.13}{##1}}}
\@namedef{PY@tok@sh}{\def\PY@tc##1{\textcolor[rgb]{0.73,0.13,0.13}{##1}}}
\@namedef{PY@tok@s1}{\def\PY@tc##1{\textcolor[rgb]{0.73,0.13,0.13}{##1}}}
\@namedef{PY@tok@mb}{\def\PY@tc##1{\textcolor[rgb]{0.40,0.40,0.40}{##1}}}
\@namedef{PY@tok@mf}{\def\PY@tc##1{\textcolor[rgb]{0.40,0.40,0.40}{##1}}}
\@namedef{PY@tok@mh}{\def\PY@tc##1{\textcolor[rgb]{0.40,0.40,0.40}{##1}}}
\@namedef{PY@tok@mi}{\def\PY@tc##1{\textcolor[rgb]{0.40,0.40,0.40}{##1}}}
\@namedef{PY@tok@il}{\def\PY@tc##1{\textcolor[rgb]{0.40,0.40,0.40}{##1}}}
\@namedef{PY@tok@mo}{\def\PY@tc##1{\textcolor[rgb]{0.40,0.40,0.40}{##1}}}
\@namedef{PY@tok@ch}{\let\PY@it=\textit\def\PY@tc##1{\textcolor[rgb]{0.24,0.48,0.48}{##1}}}
\@namedef{PY@tok@cm}{\let\PY@it=\textit\def\PY@tc##1{\textcolor[rgb]{0.24,0.48,0.48}{##1}}}
\@namedef{PY@tok@cpf}{\let\PY@it=\textit\def\PY@tc##1{\textcolor[rgb]{0.24,0.48,0.48}{##1}}}
\@namedef{PY@tok@c1}{\let\PY@it=\textit\def\PY@tc##1{\textcolor[rgb]{0.24,0.48,0.48}{##1}}}
\@namedef{PY@tok@cs}{\let\PY@it=\textit\def\PY@tc##1{\textcolor[rgb]{0.24,0.48,0.48}{##1}}}

\def\PYZbs{\char`\\}
\def\PYZus{\char`\_}
\def\PYZob{\char`\{}
\def\PYZcb{\char`\}}
\def\PYZca{\char`\^}
\def\PYZam{\char`\&}
\def\PYZlt{\char`\<}
\def\PYZgt{\char`\>}
\def\PYZsh{\char`\#}
\def\PYZpc{\char`\%}
\def\PYZdl{\char`\$}
\def\PYZhy{\char`\-}
\def\PYZsq{\char`\'}
\def\PYZdq{\char`\"}
\def\PYZti{\char`\~}
% for compatibility with earlier versions
\def\PYZat{@}
\def\PYZlb{[}
\def\PYZrb{]}
\makeatother


    % For linebreaks inside Verbatim environment from package fancyvrb.
    \makeatletter
        \newbox\Wrappedcontinuationbox
        \newbox\Wrappedvisiblespacebox
        \newcommand*\Wrappedvisiblespace {\textcolor{red}{\textvisiblespace}}
        \newcommand*\Wrappedcontinuationsymbol {\textcolor{red}{\llap{\tiny$\m@th\hookrightarrow$}}}
        \newcommand*\Wrappedcontinuationindent {3ex }
        \newcommand*\Wrappedafterbreak {\kern\Wrappedcontinuationindent\copy\Wrappedcontinuationbox}
        % Take advantage of the already applied Pygments mark-up to insert
        % potential linebreaks for TeX processing.
        %        {, <, #, %, $, ' and ": go to next line.
        %        _, }, ^, &, >, - and ~: stay at end of broken line.
        % Use of \textquotesingle for straight quote.
        \newcommand*\Wrappedbreaksatspecials {%
            \def\PYGZus{\discretionary{\char`\_}{\Wrappedafterbreak}{\char`\_}}%
            \def\PYGZob{\discretionary{}{\Wrappedafterbreak\char`\{}{\char`\{}}%
            \def\PYGZcb{\discretionary{\char`\}}{\Wrappedafterbreak}{\char`\}}}%
            \def\PYGZca{\discretionary{\char`\^}{\Wrappedafterbreak}{\char`\^}}%
            \def\PYGZam{\discretionary{\char`\&}{\Wrappedafterbreak}{\char`\&}}%
            \def\PYGZlt{\discretionary{}{\Wrappedafterbreak\char`\<}{\char`\<}}%
            \def\PYGZgt{\discretionary{\char`\>}{\Wrappedafterbreak}{\char`\>}}%
            \def\PYGZsh{\discretionary{}{\Wrappedafterbreak\char`\#}{\char`\#}}%
            \def\PYGZpc{\discretionary{}{\Wrappedafterbreak\char`\%}{\char`\%}}%
            \def\PYGZdl{\discretionary{}{\Wrappedafterbreak\char`\$}{\char`\$}}%
            \def\PYGZhy{\discretionary{\char`\-}{\Wrappedafterbreak}{\char`\-}}%
            \def\PYGZsq{\discretionary{}{\Wrappedafterbreak\textquotesingle}{\textquotesingle}}%
            \def\PYGZdq{\discretionary{}{\Wrappedafterbreak\char`\"}{\char`\"}}%
            \def\PYGZti{\discretionary{\char`\~}{\Wrappedafterbreak}{\char`\~}}%
        }
        % Some characters . , ; ? ! / are not pygmentized.
        % This macro makes them "active" and they will insert potential linebreaks
        \newcommand*\Wrappedbreaksatpunct {%
            \lccode`\~`\.\lowercase{\def~}{\discretionary{\hbox{\char`\.}}{\Wrappedafterbreak}{\hbox{\char`\.}}}%
            \lccode`\~`\,\lowercase{\def~}{\discretionary{\hbox{\char`\,}}{\Wrappedafterbreak}{\hbox{\char`\,}}}%
            \lccode`\~`\;\lowercase{\def~}{\discretionary{\hbox{\char`\;}}{\Wrappedafterbreak}{\hbox{\char`\;}}}%
            \lccode`\~`\:\lowercase{\def~}{\discretionary{\hbox{\char`\:}}{\Wrappedafterbreak}{\hbox{\char`\:}}}%
            \lccode`\~`\?\lowercase{\def~}{\discretionary{\hbox{\char`\?}}{\Wrappedafterbreak}{\hbox{\char`\?}}}%
            \lccode`\~`\!\lowercase{\def~}{\discretionary{\hbox{\char`\!}}{\Wrappedafterbreak}{\hbox{\char`\!}}}%
            \lccode`\~`\/\lowercase{\def~}{\discretionary{\hbox{\char`\/}}{\Wrappedafterbreak}{\hbox{\char`\/}}}%
            \catcode`\.\active
            \catcode`\,\active
            \catcode`\;\active
            \catcode`\:\active
            \catcode`\?\active
            \catcode`\!\active
            \catcode`\/\active
            \lccode`\~`\~
        }
    \makeatother

    \let\OriginalVerbatim=\Verbatim
    \makeatletter
    \renewcommand{\Verbatim}[1][1]{%
        %\parskip\z@skip
        \sbox\Wrappedcontinuationbox {\Wrappedcontinuationsymbol}%
        \sbox\Wrappedvisiblespacebox {\FV@SetupFont\Wrappedvisiblespace}%
        \def\FancyVerbFormatLine ##1{\hsize\linewidth
            \vtop{\raggedright\hyphenpenalty\z@\exhyphenpenalty\z@
                \doublehyphendemerits\z@\finalhyphendemerits\z@
                \strut ##1\strut}%
        }%
        % If the linebreak is at a space, the latter will be displayed as visible
        % space at end of first line, and a continuation symbol starts next line.
        % Stretch/shrink are however usually zero for typewriter font.
        \def\FV@Space {%
            \nobreak\hskip\z@ plus\fontdimen3\font minus\fontdimen4\font
            \discretionary{\copy\Wrappedvisiblespacebox}{\Wrappedafterbreak}
            {\kern\fontdimen2\font}%
        }%

        % Allow breaks at special characters using \PYG... macros.
        \Wrappedbreaksatspecials
        % Breaks at punctuation characters . , ; ? ! and / need catcode=\active
        \OriginalVerbatim[#1,codes*=\Wrappedbreaksatpunct]%
    }
    \makeatother

    % Exact colors from NB
    \definecolor{incolor}{HTML}{303F9F}
    \definecolor{outcolor}{HTML}{D84315}
    \definecolor{cellborder}{HTML}{CFCFCF}
    \definecolor{cellbackground}{HTML}{F7F7F7}

    % prompt
    \makeatletter
    \newcommand{\boxspacing}{\kern\kvtcb@left@rule\kern\kvtcb@boxsep}
    \makeatother
    \newcommand{\prompt}[4]{
        {\ttfamily\llap{{\color{#2}[#3]:\hspace{3pt}#4}}\vspace{-\baselineskip}}
    }
    

    
    % Prevent overflowing lines due to hard-to-break entities
    \sloppy
    % Setup hyperref package
    \hypersetup{
      breaklinks=true,  % so long urls are correctly broken across lines
      colorlinks=true,
      urlcolor=urlcolor,
      linkcolor=linkcolor,
      citecolor=citecolor,
      }
    % Slightly bigger margins than the latex defaults
    
    \geometry{verbose,tmargin=1in,bmargin=1in,lmargin=1in,rmargin=1in}
    
    

\begin{document}
    
    \maketitle
    
    

    
    \subsection{\# Generacion de libros de
codificacion}\label{generacion-de-libros-de-codificacion}

    \subsection{Diagrama de la estructura de clasificacion y entrenamiento
de vectores
cuantizados}\label{diagrama-de-la-estructura-de-clasificacion-y-entrenamiento-de-vectores-cuantizados}

\subsection{Creacion de clusters con distribuciones normales (Bajo
etiqueta)}\label{creacion-de-clusters-con-distribuciones-normales-bajo-etiqueta}

    \begin{tcolorbox}[breakable, size=fbox, boxrule=1pt, pad at break*=1mm,colback=cellbackground, colframe=cellborder]
\prompt{In}{incolor}{1}{\boxspacing}
\begin{Verbatim}[commandchars=\\\{\}]
\PY{k+kn}{import} \PY{n+nn}{numpy} \PY{k}{as} \PY{n+nn}{np}
\PY{k+kn}{import} \PY{n+nn}{math}
\PY{k+kn}{from} \PY{n+nn}{functools} \PY{k+kn}{import} \PY{n}{reduce}
\PY{k+kn}{from} \PY{n+nn}{collections} \PY{k+kn}{import} \PY{n}{defaultdict}
\PY{k+kn}{import} \PY{n+nn}{matplotlib}\PY{n+nn}{.}\PY{n+nn}{pyplot} \PY{k}{as} \PY{n+nn}{plt}
\PY{k+kn}{import} \PY{n+nn}{matplotlib}\PY{n+nn}{.}\PY{n+nn}{cm} \PY{k}{as} \PY{n+nn}{cm}
\end{Verbatim}
\end{tcolorbox}

    \begin{tcolorbox}[breakable, size=fbox, boxrule=1pt, pad at break*=1mm,colback=cellbackground, colframe=cellborder]
\prompt{In}{incolor}{12}{\boxspacing}
\begin{Verbatim}[commandchars=\\\{\}]
\PY{n}{SEEDS} \PY{o}{=} \PY{p}{[}\PY{l+m+mi}{2}\PY{p}{,} \PY{l+m+mi}{3}\PY{p}{,} \PY{l+m+mi}{5}\PY{p}{,} \PY{l+m+mi}{7}\PY{p}{,} \PY{l+m+mi}{11}\PY{p}{,} \PY{l+m+mi}{13}\PY{p}{,} \PY{l+m+mi}{17}\PY{p}{,} \PY{l+m+mi}{19}\PY{p}{]}
\PY{n}{n\PYZus{}clusters} \PY{o}{=} \PY{l+m+mi}{5}
\PY{n}{total\PYZus{}dimensions} \PY{o}{=} \PY{l+m+mi}{10}
\end{Verbatim}
\end{tcolorbox}

    \begin{tcolorbox}[breakable, size=fbox, boxrule=1pt, pad at break*=1mm,colback=cellbackground, colframe=cellborder]
\prompt{In}{incolor}{13}{\boxspacing}
\begin{Verbatim}[commandchars=\\\{\}]
\PY{c+c1}{\PYZsh{} Uso de centros encontrados aleatoriamente para usar distribuciones normales para generar los siguientes vectores}
\PY{n}{means} \PY{o}{=} \PY{p}{[}\PY{p}{]}
\PY{k}{for} \PY{n}{i} \PY{o+ow}{in} \PY{n+nb}{range}\PY{p}{(}\PY{n}{n\PYZus{}clusters}\PY{p}{)}\PY{p}{:}
    \PY{n}{means}\PY{o}{.}\PY{n}{append}\PY{p}{(}\PY{n}{np}\PY{o}{.}\PY{n}{random}\PY{o}{.}\PY{n}{uniform}\PY{p}{(}\PY{o}{\PYZhy{}}\PY{l+m+mi}{1}\PY{p}{,} \PY{l+m+mi}{1}\PY{p}{,} \PY{n}{total\PYZus{}dimensions}\PY{p}{)}\PY{p}{)}
\PY{n}{means}
\end{Verbatim}
\end{tcolorbox}

            \begin{tcolorbox}[breakable, size=fbox, boxrule=.5pt, pad at break*=1mm, opacityfill=0]
\prompt{Out}{outcolor}{13}{\boxspacing}
\begin{Verbatim}[commandchars=\\\{\}]
[array([-0.06709583, -0.88667795, -0.20221778, -0.45148203, -0.6643235 ,
        -0.46319055,  0.23664529, -0.05852873,  0.70771878,  0.49279505]),
 array([ 0.68708599,  0.20226601,  0.18341327, -0.74774812,  0.7650987 ,
        -0.72129978,  0.38652165, -0.57582887,  0.69081952, -0.24632298]),
 array([ 0.21961584,  0.32946061, -0.28203499,  0.87257955,  0.65850824,
         0.08164077,  0.7114203 , -0.84075668, -0.29299919, -0.90732728]),
 array([-0.75617772,  0.00696958,  0.07589178, -0.63077489, -0.46553668,
        -0.96833742,  0.90382793,  0.13985309, -0.9315108 , -0.77525297]),
 array([-0.96221021, -0.77645774,  0.79608056,  0.93436572,  0.77839736,
        -0.44553726,  0.82248796, -0.77371714,  0.24664329,  0.86172873])]
\end{Verbatim}
\end{tcolorbox}
        
    \begin{tcolorbox}[breakable, size=fbox, boxrule=1pt, pad at break*=1mm,colback=cellbackground, colframe=cellborder]
\prompt{In}{incolor}{14}{\boxspacing}
\begin{Verbatim}[commandchars=\\\{\}]
\PY{n}{desviation} \PY{o}{=} \PY{l+m+mf}{0.2}
\PY{n}{clusters} \PY{o}{=} \PY{p}{[}\PY{p}{]}
\PY{k}{for} \PY{n}{i}\PY{p}{,}\PY{n}{mean} \PY{o+ow}{in} \PY{n+nb}{enumerate}\PY{p}{(}\PY{n}{means}\PY{p}{)}\PY{p}{:}
    \PY{n}{np}\PY{o}{.}\PY{n}{random}\PY{o}{.}\PY{n}{seed}\PY{p}{(}\PY{n}{SEEDS}\PY{p}{[}\PY{n}{i}\PY{p}{]}\PY{p}{)}
    \PY{n}{vectors} \PY{o}{=} \PY{p}{[}\PY{p}{]}
    \PY{k}{for} \PY{n}{\PYZus{}} \PY{o+ow}{in} \PY{n+nb}{range}\PY{p}{(}\PY{n}{np}\PY{o}{.}\PY{n}{random}\PY{o}{.}\PY{n}{randint}\PY{p}{(}\PY{l+m+mi}{40}\PY{p}{,}\PY{l+m+mi}{100}\PY{p}{)}\PY{p}{)}\PY{p}{:}
        \PY{n}{vector} \PY{o}{=} \PY{p}{[}\PY{p}{]}
        \PY{k}{for} \PY{n}{dimension} \PY{o+ow}{in} \PY{n+nb}{range}\PY{p}{(}\PY{n}{total\PYZus{}dimensions}\PY{p}{)}\PY{p}{:}
            \PY{n}{vector}\PY{o}{.}\PY{n}{append}\PY{p}{(}\PY{n}{np}\PY{o}{.}\PY{n}{random}\PY{o}{.}\PY{n}{normal}\PY{p}{(}\PY{n}{loc} \PY{o}{=} \PY{n}{mean}\PY{p}{[}\PY{n}{dimension}\PY{p}{]}\PY{p}{,} \PY{n}{scale} \PY{o}{=} \PY{n}{desviation}\PY{p}{)}\PY{p}{)}

        \PY{n}{vectors}\PY{o}{.}\PY{n}{append}\PY{p}{(}\PY{n}{np}\PY{o}{.}\PY{n}{array}\PY{p}{(}\PY{n}{vector}\PY{p}{)}\PY{p}{)}

    \PY{n}{clusters}\PY{o}{.}\PY{n}{append}\PY{p}{(}\PY{n}{np}\PY{o}{.}\PY{n}{array}\PY{p}{(}\PY{n}{vectors}\PY{p}{)}\PY{p}{)}
\end{Verbatim}
\end{tcolorbox}

    \begin{tcolorbox}[breakable, size=fbox, boxrule=1pt, pad at break*=1mm,colback=cellbackground, colframe=cellborder]
\prompt{In}{incolor}{ }{\boxspacing}
\begin{Verbatim}[commandchars=\\\{\}]
\PY{k}{def} \PY{n+nf}{plot\PYZus{}2\PYZus{}dimensions}\PY{p}{(}\PY{n}{d\PYZus{}1}\PY{p}{,} \PY{n}{d\PYZus{}2}\PY{p}{,} \PY{n}{vectors}\PY{p}{)}\PY{p}{:}
    \PY{n}{n\PYZus{}clusters} \PY{o}{=} \PY{n+nb}{len}\PY{p}{(}\PY{n}{vectors}\PY{p}{)}
    \PY{n}{colors} \PY{o}{=} \PY{n}{cm}\PY{o}{.}\PY{n}{rainbow}\PY{p}{(}\PY{n}{np}\PY{o}{.}\PY{n}{linspace}\PY{p}{(}\PY{l+m+mi}{0}\PY{p}{,} \PY{l+m+mi}{1}\PY{p}{,} \PY{n}{n\PYZus{}clusters}\PY{p}{)}\PY{p}{)}

    \PY{k}{for} \PY{n}{i} \PY{o+ow}{in} \PY{n+nb}{range}\PY{p}{(}\PY{n}{n\PYZus{}clusters}\PY{p}{)}\PY{p}{:}
        \PY{n}{x} \PY{o}{=} \PY{n}{vectors}\PY{p}{[}\PY{n}{i}\PY{p}{]}\PY{p}{[}\PY{p}{:}\PY{p}{,} \PY{n}{d\PYZus{}1}\PY{p}{]}
        \PY{n}{y} \PY{o}{=} \PY{n}{vectors}\PY{p}{[}\PY{n}{i}\PY{p}{]}\PY{p}{[}\PY{p}{:}\PY{p}{,} \PY{n}{d\PYZus{}2}\PY{p}{]}
        \PY{n}{plt}\PY{o}{.}\PY{n}{scatter}\PY{p}{(}\PY{n}{x}\PY{p}{,}\PY{n}{y}\PY{p}{,} \PY{n}{color} \PY{o}{=} \PY{n}{colors}\PY{p}{[}\PY{n}{i}\PY{p}{]}\PY{p}{,} \PY{n}{label}\PY{o}{=}\PY{l+s+sa}{f}\PY{l+s+s2}{\PYZdq{}}\PY{l+s+s2}{Cluster }\PY{l+s+si}{\PYZob{}}\PY{n}{i}\PY{o}{+}\PY{l+m+mi}{1}\PY{l+s+si}{\PYZcb{}}\PY{l+s+s2}{\PYZdq{}}\PY{p}{)}

    \PY{n}{plt}\PY{o}{.}\PY{n}{title}\PY{p}{(}\PY{l+s+sa}{f}\PY{l+s+s2}{\PYZdq{}}\PY{l+s+s2}{Gradica de dimension: }\PY{l+s+si}{\PYZob{}}\PY{n}{d\PYZus{}1}\PY{l+s+si}{\PYZcb{}}\PY{l+s+s2}{ vs }\PY{l+s+si}{\PYZob{}}\PY{n}{d\PYZus{}2}\PY{l+s+si}{\PYZcb{}}\PY{l+s+s2}{\PYZdq{}}\PY{p}{)}
    \PY{n}{plt}\PY{o}{.}\PY{n}{xlabel}\PY{p}{(}\PY{l+s+sa}{f}\PY{l+s+s2}{\PYZdq{}}\PY{l+s+si}{\PYZob{}}\PY{n}{d\PYZus{}1}\PY{l+s+si}{\PYZcb{}}\PY{l+s+s2}{\PYZdq{}}\PY{p}{)}
    \PY{n}{plt}\PY{o}{.}\PY{n}{legend}\PY{p}{(}\PY{p}{)}
    \PY{n}{plt}\PY{o}{.}\PY{n}{ylabel}\PY{p}{(}\PY{l+s+sa}{f}\PY{l+s+s2}{\PYZdq{}}\PY{l+s+si}{\PYZob{}}\PY{n}{d\PYZus{}2}\PY{l+s+si}{\PYZcb{}}\PY{l+s+s2}{\PYZdq{}}\PY{p}{)}
    \PY{n}{plt}\PY{o}{.}\PY{n}{show}\PY{p}{(}\PY{p}{)}
    
\end{Verbatim}
\end{tcolorbox}

    \paragraph{Clusters generados:}\label{clusters-generados}

    \begin{tcolorbox}[breakable, size=fbox, boxrule=1pt, pad at break*=1mm,colback=cellbackground, colframe=cellborder]
\prompt{In}{incolor}{15}{\boxspacing}
\begin{Verbatim}[commandchars=\\\{\}]
\PY{n}{plot\PYZus{}2\PYZus{}dimensions}\PY{p}{(}\PY{l+m+mi}{6}\PY{p}{,} \PY{l+m+mi}{8}\PY{p}{,} \PY{n}{clusters}\PY{p}{)}
\end{Verbatim}
\end{tcolorbox}

    \begin{center}
    \adjustimage{max size={0.9\linewidth}{0.9\paperheight}}{output_8_0.png}
    \end{center}
    { \hspace*{\fill} \\}
    
    \begin{tcolorbox}[breakable, size=fbox, boxrule=1pt, pad at break*=1mm,colback=cellbackground, colframe=cellborder]
\prompt{In}{incolor}{16}{\boxspacing}
\begin{Verbatim}[commandchars=\\\{\}]
\PY{n}{plot\PYZus{}2\PYZus{}dimensions}\PY{p}{(}\PY{l+m+mi}{2}\PY{p}{,} \PY{l+m+mi}{3}\PY{p}{,} \PY{n}{clusters}\PY{p}{)}
\end{Verbatim}
\end{tcolorbox}

    \begin{center}
    \adjustimage{max size={0.9\linewidth}{0.9\paperheight}}{output_9_0.png}
    \end{center}
    { \hspace*{\fill} \\}
    
    \begin{tcolorbox}[breakable, size=fbox, boxrule=1pt, pad at break*=1mm,colback=cellbackground, colframe=cellborder]
\prompt{In}{incolor}{17}{\boxspacing}
\begin{Verbatim}[commandchars=\\\{\}]
\PY{k+kn}{import} \PY{n+nn}{seaborn} \PY{k}{as} \PY{n+nn}{sns}
\PY{k+kn}{import} \PY{n+nn}{pandas} \PY{k}{as} \PY{n+nn}{pd}

\PY{c+c1}{\PYZsh{} Supongamos que tus datos están concatenados y etiquetados}
\PY{c+c1}{\PYZsh{} Convertir a DataFrame}
\PY{n}{datos} \PY{o}{=} \PY{n}{np}\PY{o}{.}\PY{n}{vstack}\PY{p}{(}\PY{n}{clusters}\PY{p}{)}
\PY{n}{etiquetas} \PY{o}{=} \PY{n}{np}\PY{o}{.}\PY{n}{hstack}\PY{p}{(}\PY{p}{[}\PY{p}{[}\PY{n}{i}\PY{p}{]} \PY{o}{*} \PY{n+nb}{len}\PY{p}{(}\PY{n}{cluster}\PY{p}{)} \PY{k}{for} \PY{n}{i}\PY{p}{,} \PY{n}{cluster} \PY{o+ow}{in} \PY{n+nb}{enumerate}\PY{p}{(}\PY{n}{clusters}\PY{p}{)}\PY{p}{]}\PY{p}{)}
\PY{n}{df} \PY{o}{=} \PY{n}{pd}\PY{o}{.}\PY{n}{DataFrame}\PY{p}{(}\PY{n}{datos}\PY{p}{,} \PY{n}{columns}\PY{o}{=}\PY{p}{[}\PY{l+s+sa}{f}\PY{l+s+s1}{\PYZsq{}}\PY{l+s+s1}{Dim}\PY{l+s+si}{\PYZob{}}\PY{n}{d}\PY{l+s+si}{\PYZcb{}}\PY{l+s+s1}{\PYZsq{}} \PY{k}{for} \PY{n}{d} \PY{o+ow}{in} \PY{n+nb}{range}\PY{p}{(}\PY{n}{total\PYZus{}dimensions}\PY{p}{)}\PY{p}{]}\PY{p}{)}
\PY{n}{df}\PY{p}{[}\PY{l+s+s1}{\PYZsq{}}\PY{l+s+s1}{Cluster}\PY{l+s+s1}{\PYZsq{}}\PY{p}{]} \PY{o}{=} \PY{n}{etiquetas}\PY{o}{.}\PY{n}{astype}\PY{p}{(}\PY{n+nb}{str}\PY{p}{)}  \PY{c+c1}{\PYZsh{} Convertir a string para hue}

\PY{c+c1}{\PYZsh{} Crear pairplot}
\PY{n}{sns}\PY{o}{.}\PY{n}{pairplot}\PY{p}{(}\PY{n}{df}\PY{p}{,} \PY{n}{hue}\PY{o}{=}\PY{l+s+s1}{\PYZsq{}}\PY{l+s+s1}{Cluster}\PY{l+s+s1}{\PYZsq{}}\PY{p}{,} \PY{n}{palette}\PY{o}{=}\PY{l+s+s1}{\PYZsq{}}\PY{l+s+s1}{viridis}\PY{l+s+s1}{\PYZsq{}}\PY{p}{,} \PY{n}{diag\PYZus{}kind}\PY{o}{=}\PY{l+s+s1}{\PYZsq{}}\PY{l+s+s1}{hist}\PY{l+s+s1}{\PYZsq{}}\PY{p}{,} \PY{n}{markers}\PY{o}{=}\PY{l+s+s1}{\PYZsq{}}\PY{l+s+s1}{o}\PY{l+s+s1}{\PYZsq{}}\PY{p}{,} \PY{n}{plot\PYZus{}kws}\PY{o}{=}\PY{p}{\PYZob{}}\PY{l+s+s1}{\PYZsq{}}\PY{l+s+s1}{alpha}\PY{l+s+s1}{\PYZsq{}}\PY{p}{:}\PY{l+m+mf}{0.6}\PY{p}{,} \PY{l+s+s1}{\PYZsq{}}\PY{l+s+s1}{edgecolor}\PY{l+s+s1}{\PYZsq{}}\PY{p}{:}\PY{l+s+s1}{\PYZsq{}}\PY{l+s+s1}{k}\PY{l+s+s1}{\PYZsq{}}\PY{p}{\PYZcb{}}\PY{p}{)}
\PY{n}{plt}\PY{o}{.}\PY{n}{suptitle}\PY{p}{(}\PY{l+s+s1}{\PYZsq{}}\PY{l+s+s1}{Pairplot de Todas las Dimensiones con Seaborn}\PY{l+s+s1}{\PYZsq{}}\PY{p}{,} \PY{n}{y}\PY{o}{=}\PY{l+m+mf}{1.02}\PY{p}{)}
\PY{n}{plt}\PY{o}{.}\PY{n}{show}\PY{p}{(}\PY{p}{)}
\end{Verbatim}
\end{tcolorbox}

    \begin{center}
    \adjustimage{max size={0.9\linewidth}{0.9\paperheight}}{output_10_0.png}
    \end{center}
    { \hspace*{\fill} \\}
    
    \subsection{Creacion de libros de
codificacion}\label{creacion-de-libros-de-codificacion}

\subsubsection{Funcion de entrenamiento (Algoritmo
LBG):}\label{funcion-de-entrenamiento-algoritmo-lbg}

El algoritmo LBG (Linde-Buzo-Gray) es un método fundamental en el campo
de la cuantización vectorial, ampliamente utilizado en compresión de
señales y procesamiento de imágenes. Fue desarrollado por David Linde,
Peter Buzo y Robert Gray en 1980 y se ha convertido en una técnica
estándar para diseñar códigos de cuántización eficientes.

\paragraph{¿Qué es la Cuantización
Vectorial?}\label{quuxe9-es-la-cuantizaciuxf3n-vectorial}

Antes de profundizar en el algoritmo LBG, es importante entender qué es
la cuantización vectorial:

Cuantización: Proceso de mapear un conjunto continuo de valores a un
conjunto finito de niveles discretos. Es esencial en la compresión de
datos para reducir la cantidad de información necesaria para representar
una señal.

Cuantización Vectorial: Extiende la cuantización a vectores en lugar de
valores escalares. Esto permite capturar mejor las correlaciones entre
múltiples dimensiones de los datos, mejorando la eficiencia de la
compresión.

\paragraph{Objetivo del Algoritmo LBG}\label{objetivo-del-algoritmo-lbg}

El principal objetivo del algoritmo LBG es diseñar un código de
cuántización (un conjunto de vectores de referencia llamados
``códigos'') que minimice el error de cuantización para un conjunto dado
de datos. Es decir, busca encontrar los mejores representantes para
agrupar los datos de manera que la distancia (error) entre los datos y
sus representantes sea la menor posible.

\paragraph{Seudocodigo}\label{seudocodigo}

Fuente: https://github.com/internaut/py-lbg

    \begin{tcolorbox}[breakable, size=fbox, boxrule=1pt, pad at break*=1mm,colback=cellbackground, colframe=cellborder]
\prompt{In}{incolor}{18}{\boxspacing}
\begin{Verbatim}[commandchars=\\\{\}]
\PY{k}{def} \PY{n+nf}{find\PYZus{}centroids}\PY{p}{(}\PY{n}{vecs}\PY{p}{,} \PY{n}{dim}\PY{o}{=}\PY{k+kc}{None}\PY{p}{,} \PY{n}{size}\PY{o}{=}\PY{k+kc}{None}\PY{p}{)}\PY{p}{:}
\PY{+w}{    }\PY{l+s+sd}{\PYZdq{}\PYZdq{}\PYZdq{}}
\PY{l+s+sd}{    Calculcate average vector (center vector) for input vectors \PYZlt{}vecs\PYZgt{}.}
\PY{l+s+sd}{    :param vecs: input vectors}
\PY{l+s+sd}{    :param dim: dimension of \PYZlt{}vecs\PYZgt{} if it was already calculated}
\PY{l+s+sd}{    :param size: size of \PYZlt{}vecs\PYZgt{} if it was already calculated}
\PY{l+s+sd}{    :return average vector (center vector) for input vectors \PYZlt{}vecs\PYZgt{}}
\PY{l+s+sd}{    \PYZdq{}\PYZdq{}\PYZdq{}}
    \PY{n}{size} \PY{o}{=} \PY{n}{size} \PY{o+ow}{or} \PY{n+nb}{len}\PY{p}{(}\PY{n}{vecs}\PY{p}{)}
    \PY{n}{dim} \PY{o}{=} \PY{n}{dim} \PY{o+ow}{or} \PY{n+nb}{len}\PY{p}{(}\PY{n}{vecs}\PY{p}{[}\PY{l+m+mi}{0}\PY{p}{]}\PY{p}{)}
    \PY{n}{avg\PYZus{}vec} \PY{o}{=} \PY{p}{[}\PY{l+m+mf}{0.0}\PY{p}{]} \PY{o}{*} \PY{n}{dim}
    \PY{k}{for} \PY{n}{vec} \PY{o+ow}{in} \PY{n}{vecs}\PY{p}{:}
        \PY{k}{for} \PY{n}{i}\PY{p}{,} \PY{n}{x} \PY{o+ow}{in} \PY{n+nb}{enumerate}\PY{p}{(}\PY{n}{vec}\PY{p}{)}\PY{p}{:}
            \PY{n}{avg\PYZus{}vec}\PY{p}{[}\PY{n}{i}\PY{p}{]} \PY{o}{+}\PY{o}{=} \PY{n}{x} \PY{o}{/} \PY{n}{size}

    \PY{k}{return} \PY{n}{avg\PYZus{}vec}
\PY{k}{def} \PY{n+nf}{new\PYZus{}codevector}\PY{p}{(}\PY{n}{c}\PY{p}{,} \PY{n}{e}\PY{p}{)}\PY{p}{:}
\PY{+w}{    }\PY{l+s+sd}{\PYZdq{}\PYZdq{}\PYZdq{}}
\PY{l+s+sd}{    Create a new codevector based on \PYZlt{}c\PYZgt{} but moved by factor \PYZlt{}e\PYZgt{}}
\PY{l+s+sd}{    :param c: base codevector}
\PY{l+s+sd}{    :param e: move factor}
\PY{l+s+sd}{    :return new codevector}
\PY{l+s+sd}{    \PYZdq{}\PYZdq{}\PYZdq{}}
    \PY{k}{return} \PY{p}{[}\PY{n}{x} \PY{o}{*} \PY{p}{(}\PY{l+m+mf}{1.0} \PY{o}{+} \PY{n}{e}\PY{p}{)} \PY{k}{for} \PY{n}{x} \PY{o+ow}{in} \PY{n}{c}\PY{p}{]}


\PY{k}{def} \PY{n+nf}{avg\PYZus{}distortion\PYZus{}c0}\PY{p}{(}\PY{n}{c0}\PY{p}{,} \PY{n}{data}\PY{p}{,} \PY{n}{size}\PY{o}{=}\PY{k+kc}{None}\PY{p}{)}\PY{p}{:}
\PY{+w}{    }\PY{l+s+sd}{\PYZdq{}\PYZdq{}\PYZdq{}}
\PY{l+s+sd}{    Average distortion of \PYZlt{}c0\PYZgt{} in relation to \PYZlt{}data\PYZgt{} (i.e. how good does \PYZlt{}c0\PYZgt{} describe \PYZlt{}data\PYZgt{}?).}
\PY{l+s+sd}{    :param c0: comparison vector}
\PY{l+s+sd}{    :param data: sample data}
\PY{l+s+sd}{    :param size: size of \PYZlt{}data\PYZgt{} if it was already calculated}
\PY{l+s+sd}{    :return average distortion}
\PY{l+s+sd}{    \PYZdq{}\PYZdq{}\PYZdq{}}
    \PY{n}{size} \PY{o}{=} \PY{n}{size} \PY{o+ow}{or} \PY{n}{\PYZus{}size\PYZus{}data}
    \PY{k}{return} \PY{n}{reduce}\PY{p}{(}\PY{k}{lambda} \PY{n}{s}\PY{p}{,} \PY{n}{d}\PY{p}{:}  \PY{n}{s} \PY{o}{+} \PY{n}{d} \PY{o}{/} \PY{n}{size}\PY{p}{,}
                  \PY{p}{(}\PY{n}{euclid\PYZus{}squared}\PY{p}{(}\PY{n}{c0}\PY{p}{,} \PY{n}{vec}\PY{p}{)}
                   \PY{k}{for} \PY{n}{vec} \PY{o+ow}{in} \PY{n}{data}\PY{p}{)}\PY{p}{,}
                  \PY{l+m+mf}{0.0}\PY{p}{)}


\PY{k}{def} \PY{n+nf}{avg\PYZus{}distortion\PYZus{}c\PYZus{}list}\PY{p}{(}\PY{n}{c\PYZus{}list}\PY{p}{,} \PY{n}{data}\PY{p}{,} \PY{n}{size}\PY{o}{=}\PY{k+kc}{None}\PY{p}{)}\PY{p}{:}
\PY{+w}{    }\PY{l+s+sd}{\PYZdq{}\PYZdq{}\PYZdq{}}
\PY{l+s+sd}{    Average distortion between input samples \PYZlt{}data\PYZgt{} and a list \PYZlt{}c\PYZus{}list\PYZgt{} that contains a codevector for each point in}
\PY{l+s+sd}{    \PYZlt{}data\PYZgt{}.}
\PY{l+s+sd}{    :param c\PYZus{}list: list that contains a codevector for each point in \PYZlt{}data\PYZgt{}}
\PY{l+s+sd}{    :param data: input samples}
\PY{l+s+sd}{    :param size: Size of \PYZlt{}data\PYZgt{} if it was already calculated}
\PY{l+s+sd}{    :return:}
\PY{l+s+sd}{    \PYZdq{}\PYZdq{}\PYZdq{}}
    \PY{n}{size} \PY{o}{=} \PY{n}{size} \PY{o+ow}{or} \PY{n}{\PYZus{}size\PYZus{}data}
    \PY{k}{return} \PY{n}{reduce}\PY{p}{(}\PY{k}{lambda} \PY{n}{s}\PY{p}{,} \PY{n}{d}\PY{p}{:}  \PY{n}{s} \PY{o}{+} \PY{n}{d} \PY{o}{/} \PY{n}{size}\PY{p}{,}
                  \PY{p}{(}\PY{n}{euclid\PYZus{}squared}\PY{p}{(}\PY{n}{c\PYZus{}i}\PY{p}{,} \PY{n}{data}\PY{p}{[}\PY{n}{i}\PY{p}{]}\PY{p}{)}
                   \PY{k}{for} \PY{n}{i}\PY{p}{,} \PY{n}{c\PYZus{}i} \PY{o+ow}{in} \PY{n+nb}{enumerate}\PY{p}{(}\PY{n}{c\PYZus{}list}\PY{p}{)}\PY{p}{)}\PY{p}{,}
                  \PY{l+m+mf}{0.0}\PY{p}{)}


\PY{k}{def} \PY{n+nf}{euclid\PYZus{}squared}\PY{p}{(}\PY{n}{a}\PY{p}{,} \PY{n}{b}\PY{p}{)}\PY{p}{:}
    \PY{k}{return} \PY{n+nb}{sum}\PY{p}{(}\PY{p}{(}\PY{n}{x\PYZus{}a} \PY{o}{\PYZhy{}} \PY{n}{x\PYZus{}b}\PY{p}{)} \PY{o}{*}\PY{o}{*} \PY{l+m+mi}{2} \PY{k}{for} \PY{n}{x\PYZus{}a}\PY{p}{,} \PY{n}{x\PYZus{}b} \PY{o+ow}{in} \PY{n+nb}{zip}\PY{p}{(}\PY{n}{a}\PY{p}{,} \PY{n}{b}\PY{p}{)}\PY{p}{)}


\PY{k}{def} \PY{n+nf}{split\PYZus{}codebook}\PY{p}{(}\PY{n}{data}\PY{p}{,} \PY{n}{codebook}\PY{p}{,} \PY{n}{epsilon}\PY{p}{,} \PY{n}{initial\PYZus{}avg\PYZus{}dist}\PY{p}{)}\PY{p}{:}
\PY{+w}{    }\PY{l+s+sd}{\PYZdq{}\PYZdq{}\PYZdq{}}
\PY{l+s+sd}{    Split the codebook so that each codevector in the codebook is split into two.}
\PY{l+s+sd}{    :param data: input data}
\PY{l+s+sd}{    :param codebook: input codebook. its codevectors will be split into two.}
\PY{l+s+sd}{    :param epsilon: convergence value}
\PY{l+s+sd}{    :param initial\PYZus{}avg\PYZus{}dist: initial average distortion}
\PY{l+s+sd}{    :return Tuple with new codebook, codebook absolute weights and codebook relative weights}
\PY{l+s+sd}{    \PYZdq{}\PYZdq{}\PYZdq{}}

    \PY{c+c1}{\PYZsh{} split codevectors}
    \PY{n}{new\PYZus{}codevectors} \PY{o}{=} \PY{p}{[}\PY{p}{]}
    \PY{k}{for} \PY{n}{c} \PY{o+ow}{in} \PY{n}{codebook}\PY{p}{:}
        \PY{c+c1}{\PYZsh{} the new codevectors c1 and c2 will moved by epsilon and \PYZhy{}epsilon so to be apart from each other}
        \PY{n}{c1} \PY{o}{=} \PY{n}{new\PYZus{}codevector}\PY{p}{(}\PY{n}{c}\PY{p}{,} \PY{n}{epsilon}\PY{p}{)}
        \PY{n}{c2} \PY{o}{=} \PY{n}{new\PYZus{}codevector}\PY{p}{(}\PY{n}{c}\PY{p}{,} \PY{o}{\PYZhy{}}\PY{n}{epsilon}\PY{p}{)}
        \PY{n}{new\PYZus{}codevectors}\PY{o}{.}\PY{n}{extend}\PY{p}{(}\PY{p}{(}\PY{n}{c1}\PY{p}{,} \PY{n}{c2}\PY{p}{)}\PY{p}{)}

    \PY{n}{codebook} \PY{o}{=} \PY{n}{new\PYZus{}codevectors}
    \PY{n}{len\PYZus{}codebook} \PY{o}{=} \PY{n+nb}{len}\PY{p}{(}\PY{n}{codebook}\PY{p}{)}
    \PY{n}{abs\PYZus{}weights} \PY{o}{=} \PY{p}{[}\PY{l+m+mi}{0}\PY{p}{]} \PY{o}{*} \PY{n}{len\PYZus{}codebook}
    \PY{n}{rel\PYZus{}weights} \PY{o}{=} \PY{p}{[}\PY{l+m+mf}{0.0}\PY{p}{]} \PY{o}{*} \PY{n}{len\PYZus{}codebook}

    \PY{c+c1}{\PYZsh{} print(\PYZsq{}\PYZgt{} splitting to size\PYZsq{}, len\PYZus{}codebook)}

    \PY{c+c1}{\PYZsh{} try to reach a convergence by minimizing the average distortion. this is done by moving the codevectors step by}
    \PY{c+c1}{\PYZsh{} step to the center of the points in their proximity}
    \PY{n}{avg\PYZus{}dist} \PY{o}{=} \PY{l+m+mi}{0}
    \PY{n}{err} \PY{o}{=} \PY{n}{epsilon} \PY{o}{+} \PY{l+m+mi}{1}
    \PY{n}{num\PYZus{}iter} \PY{o}{=} \PY{l+m+mi}{0}
    \PY{k}{while} \PY{n}{err} \PY{o}{\PYZgt{}} \PY{n}{epsilon}\PY{p}{:}
        \PY{c+c1}{\PYZsh{} find closest codevectors for each vector in data (find the proximity of each codevector)}
        \PY{n}{closest\PYZus{}c\PYZus{}list} \PY{o}{=} \PY{p}{[}\PY{k+kc}{None}\PY{p}{]} \PY{o}{*} \PY{n}{\PYZus{}size\PYZus{}data}    \PY{c+c1}{\PYZsh{} list that contains the nearest codevector for each input data vector}
        \PY{n}{vecs\PYZus{}near\PYZus{}c} \PY{o}{=} \PY{n}{defaultdict}\PY{p}{(}\PY{n+nb}{list}\PY{p}{)}         \PY{c+c1}{\PYZsh{} list with codevector index \PYZhy{}\PYZgt{} input data vector mapping}
        \PY{n}{vec\PYZus{}idxs\PYZus{}near\PYZus{}c} \PY{o}{=} \PY{n}{defaultdict}\PY{p}{(}\PY{n+nb}{list}\PY{p}{)}     \PY{c+c1}{\PYZsh{} list with codevector index \PYZhy{}\PYZgt{} input data index mapping}
        \PY{k}{for} \PY{n}{i}\PY{p}{,} \PY{n}{vec} \PY{o+ow}{in} \PY{n+nb}{enumerate}\PY{p}{(}\PY{n}{data}\PY{p}{)}\PY{p}{:}  \PY{c+c1}{\PYZsh{} for each input vector}
            \PY{n}{min\PYZus{}dist} \PY{o}{=} \PY{k+kc}{None}
            \PY{n}{closest\PYZus{}c\PYZus{}index} \PY{o}{=} \PY{k+kc}{None}
            \PY{k}{for} \PY{n}{i\PYZus{}c}\PY{p}{,} \PY{n}{c} \PY{o+ow}{in} \PY{n+nb}{enumerate}\PY{p}{(}\PY{n}{codebook}\PY{p}{)}\PY{p}{:}  \PY{c+c1}{\PYZsh{} for each codevector}
                \PY{n}{d} \PY{o}{=} \PY{n}{euclid\PYZus{}squared}\PY{p}{(}\PY{n}{vec}\PY{p}{,} \PY{n}{c}\PY{p}{)}
                \PY{k}{if} \PY{n}{min\PYZus{}dist} \PY{o+ow}{is} \PY{k+kc}{None} \PY{o+ow}{or} \PY{n}{d} \PY{o}{\PYZlt{}} \PY{n}{min\PYZus{}dist}\PY{p}{:}    \PY{c+c1}{\PYZsh{} found new closest codevector}
                    \PY{n}{min\PYZus{}dist} \PY{o}{=} \PY{n}{d}
                    \PY{n}{closest\PYZus{}c\PYZus{}list}\PY{p}{[}\PY{n}{i}\PY{p}{]} \PY{o}{=} \PY{n}{c}
                    \PY{n}{closest\PYZus{}c\PYZus{}index} \PY{o}{=} \PY{n}{i\PYZus{}c}
            \PY{n}{vecs\PYZus{}near\PYZus{}c}\PY{p}{[}\PY{n}{closest\PYZus{}c\PYZus{}index}\PY{p}{]}\PY{o}{.}\PY{n}{append}\PY{p}{(}\PY{n}{vec}\PY{p}{)}
            \PY{n}{vec\PYZus{}idxs\PYZus{}near\PYZus{}c}\PY{p}{[}\PY{n}{closest\PYZus{}c\PYZus{}index}\PY{p}{]}\PY{o}{.}\PY{n}{append}\PY{p}{(}\PY{n}{i}\PY{p}{)}

        \PY{c+c1}{\PYZsh{} update codebook: recalculate each codevector so that it sits in the center of the points in their proximity}
        \PY{k}{for} \PY{n}{i\PYZus{}c} \PY{o+ow}{in} \PY{n+nb}{range}\PY{p}{(}\PY{n}{len\PYZus{}codebook}\PY{p}{)}\PY{p}{:} \PY{c+c1}{\PYZsh{} for each codevector index}
            \PY{n}{vecs} \PY{o}{=} \PY{n}{vecs\PYZus{}near\PYZus{}c}\PY{o}{.}\PY{n}{get}\PY{p}{(}\PY{n}{i\PYZus{}c}\PY{p}{)} \PY{o+ow}{or} \PY{p}{[}\PY{p}{]}   \PY{c+c1}{\PYZsh{} get its proximity input vectors}
            \PY{n}{num\PYZus{}vecs\PYZus{}near\PYZus{}c} \PY{o}{=} \PY{n+nb}{len}\PY{p}{(}\PY{n}{vecs}\PY{p}{)}
            \PY{k}{if} \PY{n}{num\PYZus{}vecs\PYZus{}near\PYZus{}c} \PY{o}{\PYZgt{}} \PY{l+m+mi}{0}\PY{p}{:}
                \PY{n}{new\PYZus{}c} \PY{o}{=} \PY{n}{find\PYZus{}centroids}\PY{p}{(}\PY{n}{vecs}\PY{p}{,} \PY{n}{\PYZus{}dim}\PY{p}{)}     \PY{c+c1}{\PYZsh{} calculate the new center}
                \PY{n}{codebook}\PY{p}{[}\PY{n}{i\PYZus{}c}\PY{p}{]} \PY{o}{=} \PY{n}{new\PYZus{}c}                   \PY{c+c1}{\PYZsh{} update in codebook}
                \PY{k}{for} \PY{n}{i} \PY{o+ow}{in} \PY{n}{vec\PYZus{}idxs\PYZus{}near\PYZus{}c}\PY{p}{[}\PY{n}{i\PYZus{}c}\PY{p}{]}\PY{p}{:}          \PY{c+c1}{\PYZsh{} update in input vector index \PYZhy{}\PYZgt{} codevector mapping list}
                    \PY{n}{closest\PYZus{}c\PYZus{}list}\PY{p}{[}\PY{n}{i}\PY{p}{]} \PY{o}{=} \PY{n}{new\PYZus{}c}

                \PY{c+c1}{\PYZsh{} update the weights}
                \PY{n}{abs\PYZus{}weights}\PY{p}{[}\PY{n}{i\PYZus{}c}\PY{p}{]} \PY{o}{=} \PY{n}{num\PYZus{}vecs\PYZus{}near\PYZus{}c}
                \PY{n}{rel\PYZus{}weights}\PY{p}{[}\PY{n}{i\PYZus{}c}\PY{p}{]} \PY{o}{=} \PY{n}{num\PYZus{}vecs\PYZus{}near\PYZus{}c} \PY{o}{/} \PY{n}{\PYZus{}size\PYZus{}data}

        \PY{c+c1}{\PYZsh{} recalculate average distortion value}
        \PY{n}{prev\PYZus{}avg\PYZus{}dist} \PY{o}{=} \PY{n}{avg\PYZus{}dist} \PY{k}{if} \PY{n}{avg\PYZus{}dist} \PY{o}{\PYZgt{}} \PY{l+m+mi}{0} \PY{k}{else} \PY{n}{initial\PYZus{}avg\PYZus{}dist}
        \PY{n}{avg\PYZus{}dist} \PY{o}{=} \PY{n}{avg\PYZus{}distortion\PYZus{}c\PYZus{}list}\PY{p}{(}\PY{n}{closest\PYZus{}c\PYZus{}list}\PY{p}{,} \PY{n}{data}\PY{p}{)}

        \PY{c+c1}{\PYZsh{} recalculate the new error value}
        \PY{n}{err} \PY{o}{=} \PY{p}{(}\PY{n}{prev\PYZus{}avg\PYZus{}dist} \PY{o}{\PYZhy{}} \PY{n}{avg\PYZus{}dist}\PY{p}{)} \PY{o}{/} \PY{n}{prev\PYZus{}avg\PYZus{}dist}
        \PY{c+c1}{\PYZsh{} print(closest\PYZus{}c\PYZus{}list)}
        \PY{c+c1}{\PYZsh{} print(\PYZsq{}\PYZgt{} iteration\PYZsq{}, num\PYZus{}iter, \PYZsq{}avg\PYZus{}dist\PYZsq{}, avg\PYZus{}dist, \PYZsq{}prev\PYZus{}avg\PYZus{}dist\PYZsq{}, prev\PYZus{}avg\PYZus{}dist, \PYZsq{}err\PYZsq{}, err)}

        \PY{n}{num\PYZus{}iter} \PY{o}{+}\PY{o}{=} \PY{l+m+mi}{1}

    \PY{k}{return} \PY{n}{codebook}\PY{p}{,} \PY{n}{abs\PYZus{}weights}\PY{p}{,} \PY{n}{rel\PYZus{}weights}\PY{p}{,} \PY{n}{avg\PYZus{}dist}


\PY{k}{def} \PY{n+nf}{generate\PYZus{}codebook}\PY{p}{(}\PY{n}{data}\PY{p}{,} \PY{n}{size\PYZus{}codebook}\PY{p}{,} \PY{n}{epsilon}\PY{o}{=}\PY{l+m+mf}{0.00001}\PY{p}{)}\PY{p}{:}
\PY{+w}{    }\PY{l+s+sd}{\PYZdq{}\PYZdq{}\PYZdq{}}
\PY{l+s+sd}{    Generate codebook of size \PYZlt{}size\PYZus{}codebook\PYZgt{} with convergence value \PYZlt{}epsilon\PYZgt{}. Will return a tuple with the}
\PY{l+s+sd}{    generated codebook, a vector with absolute weights and a vector with relative weights (the weight denotes how many}
\PY{l+s+sd}{    vectors for \PYZlt{}data\PYZgt{} are in the proximity of the codevector.}
\PY{l+s+sd}{    :param data: input data with N k\PYZhy{}dimensional vectors}
\PY{l+s+sd}{    :param size\PYZus{}codebook: codebook size. Because the codevectors are split on each iteration, this should be a}
\PY{l+s+sd}{                          power\PYZhy{}of\PYZhy{}2\PYZhy{}value}
\PY{l+s+sd}{    :param epsilon: convergence value}
\PY{l+s+sd}{    :return tuple of: codebook of size \PYZlt{}size\PYZus{}codebook\PYZgt{}, absolute weights, relative weights}
\PY{l+s+sd}{    \PYZdq{}\PYZdq{}\PYZdq{}}
    \PY{k}{global} \PY{n}{\PYZus{}size\PYZus{}data}\PY{p}{,} \PY{n}{\PYZus{}dim}

    \PY{n}{\PYZus{}size\PYZus{}data} \PY{o}{=} \PY{n+nb}{len}\PY{p}{(}\PY{n}{data}\PY{p}{)}
    \PY{k}{assert} \PY{n}{\PYZus{}size\PYZus{}data} \PY{o}{\PYZgt{}} \PY{l+m+mi}{0}

    \PY{n}{\PYZus{}dim} \PY{o}{=} \PY{n+nb}{len}\PY{p}{(}\PY{n}{data}\PY{p}{[}\PY{l+m+mi}{0}\PY{p}{]}\PY{p}{)}
    \PY{k}{assert} \PY{n}{\PYZus{}dim} \PY{o}{\PYZgt{}} \PY{l+m+mi}{0}

    \PY{n}{codebook} \PY{o}{=} \PY{p}{[}\PY{p}{]}
    \PY{n}{codebook\PYZus{}abs\PYZus{}weights} \PY{o}{=} \PY{p}{[}\PY{n}{\PYZus{}size\PYZus{}data}\PY{p}{]}
    \PY{n}{codebook\PYZus{}rel\PYZus{}weights} \PY{o}{=} \PY{p}{[}\PY{l+m+mf}{1.0}\PY{p}{]}

    \PY{c+c1}{\PYZsh{} calculate initial codevector: average vector of whole input data}
    \PY{n}{c0} \PY{o}{=} \PY{n}{find\PYZus{}centroids}\PY{p}{(}\PY{n}{data}\PY{p}{,} \PY{n}{\PYZus{}dim}\PY{p}{,} \PY{n}{\PYZus{}size\PYZus{}data}\PY{p}{)}
    \PY{n}{codebook}\PY{o}{.}\PY{n}{append}\PY{p}{(}\PY{n}{c0}\PY{p}{)}

    \PY{c+c1}{\PYZsh{} calculate the average distortion}
    \PY{n}{avg\PYZus{}dist} \PY{o}{=} \PY{n}{avg\PYZus{}distortion\PYZus{}c0}\PY{p}{(}\PY{n}{c0}\PY{p}{,} \PY{n}{data}\PY{p}{)}

    \PY{c+c1}{\PYZsh{} split codevectors until we have have enough}
    \PY{k}{while} \PY{n+nb}{len}\PY{p}{(}\PY{n}{codebook}\PY{p}{)} \PY{o}{\PYZlt{}} \PY{n}{size\PYZus{}codebook}\PY{p}{:}
        \PY{n}{codebook}\PY{p}{,} \PY{n}{codebook\PYZus{}abs\PYZus{}weights}\PY{p}{,} \PY{n}{codebook\PYZus{}rel\PYZus{}weights}\PY{p}{,} \PY{n}{avg\PYZus{}dist} \PY{o}{=} \PY{n}{split\PYZus{}codebook}\PY{p}{(}\PY{n}{data}\PY{p}{,} \PY{n}{codebook}\PY{p}{,}
                                                                                        \PY{n}{epsilon}\PY{p}{,} \PY{n}{avg\PYZus{}dist}\PY{p}{)}

    \PY{k}{return} \PY{n}{codebook}\PY{p}{,} \PY{n}{codebook\PYZus{}abs\PYZus{}weights}\PY{p}{,} \PY{n}{codebook\PYZus{}rel\PYZus{}weights}
\end{Verbatim}
\end{tcolorbox}

    \begin{tcolorbox}[breakable, size=fbox, boxrule=1pt, pad at break*=1mm,colback=cellbackground, colframe=cellborder]
\prompt{In}{incolor}{21}{\boxspacing}
\begin{Verbatim}[commandchars=\\\{\}]
\PY{n}{vectors\PYZus{}Q} \PY{o}{=} \PY{p}{[}\PY{p}{]}
\PY{k}{for} \PY{n}{cluster} \PY{o+ow}{in} \PY{n}{clusters}\PY{p}{:}
    \PY{k}{for} \PY{n}{vector} \PY{o+ow}{in} \PY{n}{cluster}\PY{p}{:}
        \PY{n}{vectors\PYZus{}Q}\PY{o}{.}\PY{n}{append}\PY{p}{(}\PY{n}{vector}\PY{p}{)}
\end{Verbatim}
\end{tcolorbox}

    \begin{tcolorbox}[breakable, size=fbox, boxrule=1pt, pad at break*=1mm,colback=cellbackground, colframe=cellborder]
\prompt{In}{incolor}{22}{\boxspacing}
\begin{Verbatim}[commandchars=\\\{\}]
\PY{n}{codebooks}\PY{p}{,} \PY{n}{absolute\PYZus{}weights}\PY{p}{,} \PY{n}{relative\PYZus{}weights} \PY{o}{=} \PY{n}{generate\PYZus{}codebook}\PY{p}{(}\PY{n}{vectors\PYZus{}Q}\PY{p}{,} \PY{l+m+mi}{5}\PY{p}{)}
\end{Verbatim}
\end{tcolorbox}

    \paragraph{Vector mas representativo del libro de
codificacion:}\label{vector-mas-representativo-del-libro-de-codificacion}

    \begin{tcolorbox}[breakable, size=fbox, boxrule=1pt, pad at break*=1mm,colback=cellbackground, colframe=cellborder]
\prompt{In}{incolor}{23}{\boxspacing}
\begin{Verbatim}[commandchars=\\\{\}]
\PY{n}{codebooks}
\end{Verbatim}
\end{tcolorbox}

            \begin{tcolorbox}[breakable, size=fbox, boxrule=.5pt, pad at break*=1mm, opacityfill=0]
\prompt{Out}{outcolor}{23}{\boxspacing}
\begin{Verbatim}[commandchars=\\\{\}]
[[-1.038453899103117,
  -0.943537433139697,
  0.8137227992358788,
  1.0076646066895907,
  0.8557125357751818,
  -0.5020709536467565,
  0.8696509262695492,
  -0.8209067100446877,
  0.24989512798527286,
  0.8976267193667539],
 [-0.8766835962258038,
  -0.6428243612432968,
  0.738172574062504,
  0.9204855045832,
  0.7565927111715296,
  -0.3499572181552667,
  0.8376410613089509,
  -0.7323522635398505,
  0.21177256757658225,
  0.8505333793595288],
 [-0.7461095820774988,
  -0.009717021871362055,
  0.08895534223414044,
  -0.6120402042424553,
  -0.46108940238323737,
  -0.9709212546790865,
  0.9076386895925539,
  0.1347472909094469,
  -0.9579751699611747,
  -0.7745613983756467],
 [0.19033295480838697,
  0.32230827992685157,
  -0.257624149673391,
  0.8903903390786898,
  0.6951884950049176,
  0.10735410104956968,
  0.7291246166000324,
  -0.8291652205829402,
  -0.2598623688647719,
  -0.9479767048898765],
 [0.7810098273353736,
  0.146893377027291,
  0.1494241586442615,
  -0.8260748587831941,
  0.8318980173501869,
  -0.8016191218484492,
  0.34306125253724995,
  -0.6361397095129382,
  0.8093688856681183,
  -0.27601536199842136],
 [0.663605197160794,
  0.2604877488697428,
  0.20338180926756877,
  -0.6884096951888655,
  0.7219505608040772,
  -0.6305913717873324,
  0.4390302830360097,
  -0.5043563465015516,
  0.5650383831665978,
  -0.2199976748963687],
 [0.017372870889339583,
  -0.8050422810985278,
  -0.2349906411932054,
  -0.5753972989014573,
  -0.8307581561352313,
  -0.4597471731566003,
  0.30157557538477464,
  -0.18145233869247243,
  0.6959404510362701,
  0.4911750785883158],
 [-0.017664232910523187,
  -0.8917483209903059,
  -0.19625645349481047,
  -0.32213363775885584,
  -0.5496050842645845,
  -0.5022462249209512,
  0.20508463800687798,
  -0.01893900356710832,
  0.7167199312436136,
  0.4037204163176102]]
\end{Verbatim}
\end{tcolorbox}
        
    \begin{tcolorbox}[breakable, size=fbox, boxrule=1pt, pad at break*=1mm,colback=cellbackground, colframe=cellborder]
\prompt{In}{incolor}{24}{\boxspacing}
\begin{Verbatim}[commandchars=\\\{\}]
\PY{n}{relative\PYZus{}weights}
\end{Verbatim}
\end{tcolorbox}

            \begin{tcolorbox}[breakable, size=fbox, boxrule=.5pt, pad at break*=1mm, opacityfill=0]
\prompt{Out}{outcolor}{24}{\boxspacing}
\begin{Verbatim}[commandchars=\\\{\}]
[0.10025706940874037,
 0.06683804627249357,
 0.2236503856041131,
 0.1928020565552699,
 0.10282776349614396,
 0.10796915167095116,
 0.10282776349614396,
 0.10282776349614396]
\end{Verbatim}
\end{tcolorbox}
        
    \begin{tcolorbox}[breakable, size=fbox, boxrule=1pt, pad at break*=1mm,colback=cellbackground, colframe=cellborder]
\prompt{In}{incolor}{25}{\boxspacing}
\begin{Verbatim}[commandchars=\\\{\}]
\PY{n}{absolute\PYZus{}weights}
\end{Verbatim}
\end{tcolorbox}

            \begin{tcolorbox}[breakable, size=fbox, boxrule=.5pt, pad at break*=1mm, opacityfill=0]
\prompt{Out}{outcolor}{25}{\boxspacing}
\begin{Verbatim}[commandchars=\\\{\}]
[39, 26, 87, 75, 40, 42, 40, 40]
\end{Verbatim}
\end{tcolorbox}
        
    Son 6 vectores debido a que los clusters de LBG son pares por el factor
de division en dos

    \begin{tcolorbox}[breakable, size=fbox, boxrule=1pt, pad at break*=1mm,colback=cellbackground, colframe=cellborder]
\prompt{In}{incolor}{ }{\boxspacing}
\begin{Verbatim}[commandchars=\\\{\}]

\end{Verbatim}
\end{tcolorbox}


    % Add a bibliography block to the postdoc
    
    
    
\end{document}
